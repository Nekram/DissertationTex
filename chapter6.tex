\chapteruaf{Using Page Access Timings Subject to Kernel SamePage Merging to Detect VMI}

\section{Motivation and Introduction}

In this experiment we leverage the memory de- duplication mechanism available in Linux called Kernel Same-Page Merging (KSM) ~\cite{arcangeli_increasing_2009}. KSM works by scanning the running Linux processes and marking those pages which produces identical hashes as candidates for merging. Those which are marked as candidates are then checked byte by byte to ensure they are indeed identical. The page table entries for the merged copies are all switched to point to only one of the previous pages. The other pages are then marked as being reclaimable by the OS. A copy-on-write scheme is employed in KSM much the same as it is in ESXi. 


Since KVM is part of the Linux kernel ~\cite{_Linux_archive} it leverages much of the existing kernel code to peform such tasks as scheduling and storage management. As a result VMs being run on a KVM hypervisor are subject to memory de-duplication with others VMs as well as processes running on the host. 


Since a VMI agent and a VM running on the same physical host and both are treated as processes by the Linux kernel we hypothesize that the shared memory between the two can be de-duplicated. Further if these pages have been merged then they will be subject to COW and therefore writing to these pages will be measurably slower than ordinarily writing to a page. 

\section{Experiment}

For this experiment we use the same apparatus as before. Only KVM is used for this experiment however. While Xen does employ a memory de-duplication technique it is only applicable to hardware virtualized guests. Since the dom0 VM is necessarily paravirtualized ~\cite{barham_xen_2003} and the VMI agent typically runs on the dom0 this experiment is inappropriate for use with Xen. 

As in our previous experiments this experiment is broken into two portions; the host portion and the guest portion. On the guest a monitoring process is run. This process allocates a page in memory and again fills it with random values using the mersenne twister ~\cite{matsumoto_mersenne_1998} again to ensure that the values in the page are unique with a high degree of probabil- ity. The address of the data in memory is then printed as well as the process ID (PID). The monitoring process then sleeps for 600s to ensure that the pages have ade- quate time to de-duplicate. A random value is then written to a random index of the page and the time to perform the write is recorded. This measurement is repeated 100 times and then plotted. 

On the host side the memory address for the monitoring process is mapped one time at a random interval in during the test. We repeated this 1,000 minute test 10 times. A sample from the monitoring process was taken when the guest was not monitored by VMI as well as a control sample.
