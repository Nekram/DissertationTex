\chapteruaf{Using Cache Timings to Detect VMI}

\section{Motivation and Introduction}
In this chapter we discuss a statistical analysis of the time taken to write to a page after it has been forcibly ejected from cache. When we look at resources shared between a host and guest one of the obvious ones is the CPU cache.  When a VMI agent fetches memory from a guest VM it ends up in the CPU cache. The question is can this caching be measured by the guest VM?  We hypothesize that there will be a measurable decrease in the time taken to access a page at intervals where VMI has been performed on the VM. 


\section{Experiment}

We begin this experiment with the same physical setup described in chapter II.  We pin both the VM and the VMI agent to run on the same VCPU.  This is to ensure they share the same cache. This experiment is broken into two portions: the monitoring portion and the host portion. 

When a VMI agent is used on a VM the memory being analyzed is copied or mapped from the VM to the VMI agent. This will move this data into the CPU cache of the VCPU on which the VMI agent is being run. Suppose the VMI agent were co-located on the same VCPU as the VM being monitored. This would mean that the VMI agent would be able to fetch the memory from cache rather than having to go out to main memory. Now suppose that the process inside the target VM which is being monitored by the VMI agent flushes cache lines which its memory occupies. This should cause a measurable increase in the amount of time required to access the data which has been evicted. 

For this experiment a process (the monitoring process) will be run on the guest VM. This process begins by allocating a page of memory and seeding it using the mersenne twister  algorithm~\cite{matsumoto_mersenne_1998}. The mersenne twister is chosen to increase the probability that page will be distinct from all other pages in memory. This is to ensure when the data is evicted from cache it will not be inadvertently loaded by other processes, which have the same data in memory. 

The monitoring process will then take one time stamp using the chrono object described in section 4.2, write to a random element in the page, then take another time stamp, and record the difference between the timestamps. The monitoring process then flushes the cache using the X86 $clf$ instruction ~\cite{_intel_2014}. This instruction flushes a cacheline from all levels of cache on the CPU. Since instruction only flushes a cache line the page is simply iterated through until all cache lines are flushed. This process is then repeated 1,000,000 times. 

On the host side the VMI agent will fetch the memory from that process and page. The VMI agent was run at regular intervals during the experiment. Trials were performed where the VMI agent was run every 50$\mu s$ , 100 $\mu s$ , 200 $\mu s$, 1ms, 10ms, and 100ms. Trials were also performed where the target VM was not monitored by a VMI agent. 

\section{Analysis}

The result were plotted and it was expected that noticeable drops in the time taken to write to a page would be present at regular intervals where the VMI agent had mapped the guest data. From Fig Y we can see that these drops memory access time are not present. Why is this the case? According to Hennesy and Patterson ~\cite{hennessy_computer_2012} say that a cache miss has takes approximately 25ns. This is a full order of magnitude less than the overhead taken for our timing measurements. Can anything be salvaged from these results? 

When one takes a histogram of the results 3 one can see that the histograms are slightly different. To determine whether or not this difference is significant we employ the t-test ~\cite{welch_generalization_1947} on pairs of datasets. The t-test gives us a measure of the difference between two samples called the t-statistic. By integrating the student-t distribution from the t-statistic to the p-value can be determined. The p-value gives us the probability that the t-statistic could be obtained under the null hypothesis that both samples are drawn from the same population. 

For the initial set of tests the t-test is run between pairs of populations with the control population being unmonitored by VMI and the variable population being monitored by VMI at some regular interval. Table 1 shows the results. 
In all the tests performed the p-value was less than 10−6. A p-value this small indicates that we can reject the null hypothesis that both samples were drawn from the same population. As a result we can take conclude that VMI has been detected by this probe.

In these tests 1,000,000 samples were used. This is an extremely large sample size and may not be useful for real time applications. The question now becomes can the sample size be reduced and still give acceptable results? Further how far can the sample be reduced and produce acceptable results? 

To determine this, we model the experiment as two random variables and considering the mutual information between them as discussed in chapter III. The results of the mutual information between the timing data and the presence or absence of VMI are shown in table 2, and demonstrate that by taking timing measurements, we gain as much as 0.08 bits of certainty about the VMI hypothesis. 


This indicates that the number of samples required can be reduced significantly from the original 1,000,000. We reduce the sample size to 100 samples which a 10,000 fold decrease in sample size. We perform the t-test again to determine if this reduction in sample size will still give positive results.  We can see from table <figure out table> that we are still able to deter- mine the difference between the different samples. Again all p-values computed for these tests are less than $10^−6$. 



