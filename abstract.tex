With the increased prevalence of virtualization in the modern computing environment, the security of that technology becomes of paramount importance. Virtual Machine Introspection (VMI) is one of the technologies that has emerged to provide security for virtual environments by examining and then interpreting the state of an active VM. VMI has seen use in systems administration, digital forensics, intrusion detection, and honeypots. As with any technology, VMI has both productive uses as well as harmful uses. The research presented in this dissertation has two goals; to enable a guest VM to determine if it is under examination by an external VMI agent and to subvert that VMI agent if the guest determines itself to be under examination. To determine if a VM is under examination a series of statistical analyses are performed on timing data generated by the guest itself. To subvert the VMI agent we look to see if there are any bugs in the VMI software itself. By manipulating the internal data structures of the guest VM we are able to exploit the bugs in the VMI agent and force arbitrary code execution.