\chapteruaf{Sysstat Experiment}
\section{Introduction and Motivation}
In our first chapter investigating methods of detecting VMI we first ask ``What are the shared resources we need to investigate?'' and ``How can we get the information we need from these shared resources?'' Since all physical hardware is shared between the host and the guest we have an abundance to choose from such as memory, CPU, disk, and the network. We can get this information directly from the OS itself as a fair amount of information is recorded by the OS. In this experiment we will take that information provided by the OS and analyze it to determine if it can yield information about the use of VMI on that guest. 


\section{Sysstat}

Modern OSs log a large amount of information for performance and monitoring purposes such as the page fault rate, CPU frequency, and disk IO rates. In the Linux OS a program called Sysstat ~\cite{godard_sysstat_2010} makes this information easily available to the user. In this experiment we attempt to analyze the data produced by Sysstat in order to determine if a VM is being monitored by a VMI agent. We hope the use of an extant tool like sysstat will allow easy monitoring of this field by a system administrator and will require minimal setup on their part. 

\section{Experiment}

We begin our experiment with the same apparatus as described in section ~\ref{Apparatus}. For our first step we synchronize the clocks on the host and guest. We do this using the NTP protocol ~\cite{mills_internet_1991} available in Linux. Synchronizing the clock on the host and guest allows us to compare measurements taken by Sysstat on the guest to times when a VMI agent was used by the host.


On the guest Sysstat is run to gather all data it's capable of gathering. The interval is set to 1s as it is the smallest measurement Sysstat can take. One hour of data was taken. The command used to gather the data is 

%TODO FIX LATER
%Specifically centering
\begin{center}\label{SAR}
\begin{verbatim}
	sar -A 1 3600
\end{verbatim}
\end{center}

On the host we run a script called \textit{collectData.py} (Appendix I). This script runs a VMI program specified for a user and at a time interval also specified by the user. For our experiment we run the VMI programs \textit{process-list}, \textit{module-list}, and \textit{map-addr}. We do our measurements at intervals of 100s, 50s, 25s, 10s, and 5s. Each time a measurement is made the time stamp is noted.

\section{PreProcessing}

The data produced by Sysstat is stored in a difficult to read binary format. In order to convert this into an easily readable format we use the program sadf~\cite{godard2010sysstat} which converts the binary data to a more useful format (a .csv file in our case). This file however is still somewhat difficult to read for the average csv reader as the data is not uniformly formatted.  Upon inspection of this field (insert here) is uniformly 0 in all of our measurements which allows us to exclude the data from our analysis. Further complicating matters is that the majority of the measurements taken by Sysstat are not of the same unit. This poses two problems: you cannot directly compare measurements of different units and that different measurements are often of different scales. For instance you cannot compare amperes and meters as one measures electrical current and one measures distance. Further an every day measurement of current may be on the order of $10^{-3}A$ but measurements of distance might be on the order of $1m$. So while a change of $10^3$ might be insignificant for a measurement of distance it might be very significant for measurement of current. To address this problem a common technique called Z-score is used. To compute the Z score of a data set we first split the data set into features. A feature is a type of measurement in our data set such as page faults per second. Since our data is conveniently divided into fields we will use each field as one feature. For each feature we compute the mean ($\mu$) and the standard deviation ($\sigma$). Then for each datum $x$ in the feature we compute and replace it with $Z$ ~\ref{ZScore}.


\begin{equation}\label{ZScore}
	Z = \frac{x-\mu}{\sigma}
\end{equation}


This however will not work when $\sigma$ is 0 which will occur when there is no variation inthe data at all. When this occurred we were able to inspect the features which were uniformly which allowed us to remove the feature entirely. 

After the features were preprocessed we matched them with the times that the VM was monitored by VMI as noted by \textit{collectData.py}. We compare the time stamps taken by sysstat to the ones taken by \textit{collectData.py}. Data points which are within $0.5s$ of the time noted by the VMI program are marked as monitored by VMI and denoted with a 1. Other points are marked as unmonitored which is denoted by a 0. 

After processing the data we still had more than 150 features available each with 3600 measurements which can be quite a large dataset when using machine learning algorithms which can be quite slow. To do this we employ feature selection.


\section{Analysis}
\subsection{Information Gains}
Suppose we have a dataset $X$ with $x_i$ samples of class $i$ and $m$ classes total (in our case two for monitored and unmonitored). The amount of information needed to classify a sample is given by 

\begin{equation}\label{InfoGain}
	I(x_1,...,x_m)=-\sum_{i=1}^{m}\frac{x_i}{x}log_2 \frac{x_i}{x}
\end{equation}

Now let us denote a feature by $F$. A feature $F$ is made up of $\nu$ subsets $\{x_1,x_2,...,x_\nu \}$ where $x_j$ is the subset of $F$ with the value $f_\nu$. Now we let $x_j$ contain $x_{ij}$ samples of class $i$. We can then compute the entropy of the feature with equation ~\ref{Entropy} 

\begin{equation}\label{Entropy}
	E(F) = \sum_{j=1}^{\nu} \frac{x_{1j}+x_{2j}+...+x_{mj}}{s}I(x_1,...,x_m)
\end{equation}

The information gain is then computed as 

\begin{equation}\label{Gain}
	Gain(F)=I(x_1,...,x_m)-E(F)
\end{equation}

Using the information gain we are able to select the features which contribute the most information to classifying the datum. 

\subsection{Weka}
For our classifications rather than using the \textit{Scikit-learn} as we did with our previous chapters we will be using the machine learning tool Weka ~\cite{hall2009weka}. Weka was chosen as it has a graphical interface which lets us switch between machine learning methods for rapid prototyping. 